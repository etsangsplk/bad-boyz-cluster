\chapter{System Comparisons}
The following table compares each of the cloud computing services explored in the previous sections.

\begin{table}[h]\footnotesize
\centering
\hyphenpenalty=5000
\tolerance=1000
\begin{tabular}{|p{4cm}||p{2.5cm} p{2.5cm} p{2.5cm} p{2.5cm}|}
\hline 
\tebf{Properties} & \tebf{Google App Engine} & \tebf{Microsoft Azure} & \tebf{Amazon EC2} & \tebf{Aneka}\\
\hline 
\hline 
\tebf{Service Type} & \te{PaaS \& IaaS} & \te{PaaS \& IaaS} & \te{IaaS} & \te{PaaS} \\
\hline
\tebf{Supported Services} & \te{Deploy (Web Applications)} & \te{Deploy/Storage} & \te{Deploy/Storage} & \te{Deploy} \\
\hline
\tebf{Deployment} & \te{Web Applications} & \te{Azure Services} & \te{Customisable VM} & \te{Applications} \\
\hline
\tebf{Scaling} & \te{Automatic} & \te{Automatic} & \te{Manual} & \te{Manual} \\
\hline
\tebf{Abstraction \mbox{of Parallelism}} & \te{Full} & \te{Full} & \te{None} & \te{Some}\\
\hline
\tebf{Deploy on third party infrastructure?} & \te{No} & \te{No} & \te{No} & \te{Yes} \\
\hline
\tebf{Page Delivery Time\ftSCspeed{} (seconds)} & \te{7.307} & \te{8.039} & \te{9.849} & \te{System \mbox{Dependent}} \\
\hline
\end{tabular}
\caption{Comparison of each of the cloud computing services}
\end{table}
\ftSCspeedText

Each system covered offers different services to the user, and as such each has strengths and weaknesses based on what is required from the Cloud service. 

Both Google App Engine and Windows Azure provide not only the infrastructure required to run scalable applications, but they additionally provide a platform for application development designed to best take advantage of the infrastructures they provide. This allows for very efficient use of resources, and for the easier development of applications, at the cost of inflexibility in the type of applications that can be deployed and tying applications to the infrastructure of the platform being used. Google App Engine and Microsoft Azure differ slightly in the type of applications they are best suited to. Google App Engine is directed toward the development and deployment of Web Applications, while Microsoft Azure is aimed to deploy Windows based applications into a scalable cloud environment.

Amazon EC2 differs from Google App Engine and Microsoft Azure in that it doesn't provide a Platform service on top of its infrastructure. Through the use of Virtual Machines (VMs), Amazon EC2 is remarkably flexible in what type of applications can be developed on it, at the cost of the inbuilt scalability and abstraction provided by the Platforms as part of Google App Engine and Microsoft Azure. It is therefore up to the user to ensure that their applications can take best advantage of the infrastructure provided by Amazon EC2, making application development more difficult. A key difference in the infrastructure made available by Amazon EC2 is the ability to select the location of your instances, giving you flexibility to place your instances nearer to where your target audience is located. It is however, as a result of all this flexibility, more complex to configure and set up than other options, and poor inter-node performance limits can limit its usefulness in applications that are not simply parallelised.  

Manjrasoft's Aneka aims to try and provide a compromise between the two by acting purely as a PaaS. It provides a level of abstraction which allows for easier development of scalable applications, while additionally being platform agnostic. This allows it to take advantage of wasted CPU cycles on a series of office desktops, or seamlessly scale to make use of infrastructure services such as Amazon EC2 when extra resources are required. It allows for the flexibility in both what infrastructure to use, as well as flexibility in the applications that can be built and developed. 

The downside of Manjrasoft's Aneka is that your applications are still tied to the Platform as with the other PaaS solutions. As the platform service is not tied to a specific infrastructure, not as much can be done to get the most out of the infrastructure as can be done when the platform and infrastructure are tightly interwoven. Aneka is also less available than the other services covered. Amazon EC2, Google App Engine and Windows Azure can be signed up for through their respective websites, however Aneka requires (beyond its trial period) contacting Manjrasoft to determine future billing arrangements.
