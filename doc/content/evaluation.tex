\chapter{Evaluation}
The Grid allows for the sending of self-contained executables across a heterogeneous grid. It suffers from a few key weaknesses. The first of these is the insecure method of communication between The Master, The Client, and The Nodes. In a distributed computing environment, a lot of uses are confidential and as such the transfer of files to remote computers needs to be secured from outside interference and interception. One way in which this could be resolved is to use SSL in order to take advantage of the HTTP based transfer and authentication layer used in The Grid.

It also has the weakness that executables must be entirely self contained. Without being entirely self contained, a executable may look for linked system libraries which will not be present on the distributed nodes. Any compiler that uses architecture specific compilations may also suffer as these executables may not operate on all Nodes in The Grid. It is our recommendation that users of The Grid ensure that each Node has minimum versions of common languages such Python and Java installed on all machines. Users of The Grid can then be assured that if they write their software to work with these languages, that their software will execute. 

There is also an inherent weakness in allowing users access to a remotely distributed network. Doing so puts a large amount of trust in the user, who through malicious or simply accidental means, could do quite a bit of damage to The Grid. This is always going to be a factor if you are allowing users to execute remote code in a distributed environment. To minimise this, a more complex virtual file system could be used to better sandbox and limit the functionality the submitted executables have over the system. 

The Grid also only accepts very limited software in terms of the way it operates. The only form of parallelism available is for problems that are ``Embarrassingly Parallel''. That is, an executable must be supplied with the input files, pre-split, which it can then work on independently without inter-node communication. The weakness of this method is still that the files must be pre-split, and the output files re-joined manually or by another program in order to be used by The Grid. 

While The Grid is quite simple in its current state, with a few improvements as outlined in the Future Improvements section, and outlined in this section, it could potentially become a fairly powerful tool for easily deploying distributed heterogeneous computer grids for the delivery of computation power as a resource. The basic requirements of Python in order to get started mean that it is quite simple to get running on a small internal network of desktop machines and would allow the use of their processing power to work on distributed tasks. 
