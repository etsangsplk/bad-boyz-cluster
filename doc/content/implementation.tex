\chapter{Implementation}
The Grid is implemented using Python 2.6, with a package dependency on Paste. Paste is a multithreaded HTTP server which is used as the underlying communication protocol in The Grid. Communication between The Client and The Master, and The Master and The Nodes is done via an agreed JSON based API.

Python 2.6 was chosen for both internal and external reasons. Internally, the development group all knew Python better than any other language which was suitable for the task. Secondly, development in Python is quite fast, though Python itself can be quite slow. As the main latency in The Grid is in the sending of input files and executables around, the computational speed loss due to language choice was deemed negligible. Externally, Python 2.6 was chosen due to its commonly installed nature which would make it easier to install The Node software on potential nodes.

The dependency on an external library Paste was determined nessecary, though external libraries were avoided where possible. This was done to both avoid licensing issues, as well as making the software easier to install as it needed to be portable and installed on many different environments due to the heterogenous nature of The Grid. Neither of the WSGI servers available in the standard library of Python were suitable as neither is multi-threaded, which is unworkable due to having to send possibly large files around that would
otherwise block the server from accepting new requests.

The choice of HTTP to communicate over was done so because of the familiarity of the development team with HTTP based communication protocols and the development of RESTful APIs with which to communicate via JSON over HTTP. It also allowed for the easy development of a web based GUI which both had the benefit of playing to the development teams strengths as well as allowing for a interface to The Grid that is accessible anywhere in the world via the internet. HTTP also has built in authentication which was utilised in order to provided a level of security for who can access and use The Grid. HTTP is also extendible to use HTTPS which would allow for the easy securing of communication within The Grid as a possible future extension of the software.

JSON specifically was chosen as it is light-weight and easier to parse than XML, while also being easily used from within JavaScript which allowed the web service on The Master to utilise the same RESTful API as the Client and Nodes do. 

The code was developed following the MVC style architecture, though as the majority of the views are actually JSON, the majority of the code is organised into Models and Controllers. Each of the web servers on The Master and The Nodes listen to a provided list of Routes which map URIs and request type (GET, POST, PUT, DELETE) to a controller which calls any specific functions from the model and returns the relevant output JSON.

\section{The Master}
The Master instansiates a HTTP Server on a given hostname and port, which it listens on for both communications from instances of The Client as well as web requests to the Graphical User Interface provided. The Master also uses this HTTP Server to accept messages from each of The Nodes in The Grid.

\subsection{The Scheduler}
The Master contains a Scheduler, which runs inside its own thread. It periodically polls the internal node list to see if any nodes have become available. If a node is available, it will check to see if there are any Work Units available that can be run on that node. If there is a job available, then The Scheduler will allocate the Work Unit to that node. It will then continue looking at available nodes and attempting to find available Work Units for them.

The Scheduler itself is itself contained within The Master and can be dynamically changed. In the case that a Scheduler is swapped however, the internal state unique to one Scheduler type will be lost and all jobs currently in the queue will be allocated using the new scheduling algorithm. The Scheduling algorithms available are: FCFS (First come, first served), Round Robin, Deadline First, Cost Constrained Deadline First, and Mutli-level Priority Queues.

\subsubsection{First Come First Serve}
FCFS completes each job sequentially as they arrived. Jobs are ordered by the time they were created, and all of the Work Units in a job will be completed before Work Units from another job are executed. If a job is created but is not ready (not all files have arrived on The Master) then a job that has been created later but is ready will be executed. When the earlier job becomes ready, all Work Units from the earlier job will be executed before resuming the job that was ready first but created later. This is done as to not unfairly punish jobs with large input files by making them wait until they are ready to be considered in the queue.

\subsubsection{Round Robin}
Round Robin is similar to FCFS however rather than executing all of the Work Units from a job before moving to the next one, it iterates over the scheduled jobs, allocating them one Work Unit at a time. This is achieved by storing a persistent queue of Job IDs between polls checking the internal node list. The Work Unit from the front of this queue is allocated to a free node, and then this job ID is placed at the back of the queue. New jobs which are sent to the grid are placed at the back of the queue.

\subsubsection{Earliest Deadline First}
Earliest Deadline First takes into account the deadline of the job and prioritises jobs by earliest deadline first. That is, if a job needs to be finished by tomorrow and there is another job that can be finished in a week, then the job that needs to be finsihed by tomorrow will take preference. As Jobs are allocated by schedulers as Work Units, a job may execute some work units before a Job is created that has an earlier deadline. In this event any remaining Work Units from the already running job will be placed behind the Work Units of the new scheduler. As a jobs deadline is a fixed amount of time in the past, indefinate starvation of a job cannot occur as eventually the deadline of the job will make it the highest priority to be completed. This deadline is calculated as the job's specified deadline minus its wall time, as this is the latest date the job can start. This prevents long jobs from being starved out until they can't possibly finish by shorted jobs.

\subsubsection{Cost Constrained Earliest Deadline First}
Deadline/Cost First takes into account not only the deadline of each Work Unit as in Deadline First but also takes into account the budget of the Job. It first ensures that a job only runs on nodes that are within the Jobs budget. It secondly preferentially places jobs with higher budgets about jobs with lower budgets.

\subsubsection{Multi-level Priority Queues}
This Multi-Level Priority Queue implementation splits The Nodes on The Grid into seperate queues. There can be any number of different queues, with a different proportion of The Grid allocated to them. The default queues are Default, Batch and Fast. Half of The Nodes on The Grid are allocated to Default, which is for any generic job between 1 hour and a few days. Batch is allocated 30\% of the nodes and is for jobs that will take a week or more. Fast is allocated 20\% of the nodes and is for jobs that are less than an hour. These priorities reserve nodes for specific job types without indefinately starving any one type of job if there is an excess of another.

This scheduler also stops large jobs from taking up most of the queue during offpeak periods, and then causing a large back log of other jobs. This is an unfortunate side effect of not being able to pause and resume a Work Unit once it has begun. While Deadline First and such can interupt a low priority job in terms of stopping further execution of additional work units, if a job is already running a number of work units which each may take a long time, there is no way to interrupt these running processes.

Different scheduling algorithms are used for each of the three different types of queues. These were all cost constrained to prevent jobs from running on nodes they couldnt afford. Where two jobs were equal in the algorithm, the one with the highest budget took preference. For the Batch queue, FCFS was used because it provides high throughput. The Fast queue uses Round Robin for its good response time, and the Default queue uses Earliest Deadline First.

\section{The Node}
Each Node, like The Master, also instansiates a HTTP Server on a given hostname and port. It listens on this for communications from The Master. The Master sends job information to The Node, which in turn uses this information to request from The Master any files that The Node requires in order to complete the job it has been assigned. In addition to its regular communications with The Grid, The node also checks any tasks that it is running, killing any which exceed their specified wall time or deadline.
