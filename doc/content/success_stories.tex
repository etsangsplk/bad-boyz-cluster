\chapter{Success Stories}
\section{GigaPan}
GigaPan is a image exploration service which allows users to explore, share and comment on gigapixel panorama images. Formed in 2008 as a commercial spin-off of successful collaborative research between NASA and Carnegie Mellon University (CMU), GigaPan delivers a service which allows users to explore the extraordinary detail of over 50,000 high-resolution panoramas from around the world. These panoramas are created from thousands of images taken from a robotic camera mount, and stitched together using image stitching software.

In January 2009 GigaPan took a panorama shot of the Obama Inauguration speech. Shortly afterwards this huge panorama image went viral, resulting in 30 million views in just two days. This prompted the developers to try Google App Engine removing the 200 mb/s load from the CMU network where the site was originally hosted. This had fantastic results for the developers of GigaPan. The intense demand was removed from CMU's network, and they were still able to provide the gigapixel images latency free, which was one of their original concerns. Since this trial GigaPan has moved its entire website to Google App Engine python platform and further developed tools which integrate with Google's other services such as Google Earth and Google Maps.

% Google code Developer Conference in May 2009.
% Rich Gibson, Gabrielle O'Donnel

% References:
% Interview: http://www.google.com/events/io/2009/sandbox/gigapan.html
% Usage: https://developers.google.com/appengine/casestudies#gigapan
% About: http://www.gigapan.com/cms/about-us
