\chapter{Success Stories}
\section{Google App Engine -- GigaPan}
GigaPan is a image exploration service which allows users to explore, share and comment on gigapixel panorama images. Formed in 2008 as a commercial spin-off of successful collaborative research between NASA and Carnegie Mellon University (CMU), GigaPan delivers a service which allows users to explore the extraordinary detail of over 50,000 high-resolution panoramas from around the world. These panoramas are created from thousands of images taken from a robotic camera mount, and stitched together using image stitching software\ftSgaeOne.
\ftSgaeOneText

In January 2009 GigaPan took a panorama shot of the Obama Inauguration speech. Shortly afterwards this huge panorama image went viral, resulting in 30 million views in just two days. This prompted the developers to try Google App Engine removing the 200 mb/s load from the CMU network where the site was originally hosted. This had fantastic results for the developers of GigaPan. The intense demand was removed from CMU's network, and they were still able to provide the gigapixel images latency free, which was one of their original concerns\ftSgaeTwo. Since this trial GigaPan has moved its entire website to Google App Engine python platform and further developed tools which integrate with Google's other services such as Google Earth and Google Maps\ftSgaeThree.
\ftSgaeTwoText\ftSgaeThreeText

\section{Microsoft Azure -- Sharpcloud}
Sharpcloud is a British company that was formed in 2009 on a simple but groundbreaking idea: to apply highly visual and commonly used social-networking tools to the crucial corporate project of developing long-term road maps and strategy. 

The Sharpcloud service involves executives and other users working within a Web browser, creating a framework for their road map with a single click on the interface. They then define the various attributes and properties they want to track, such as benefit, cost, and risk. The idea behind Sharpcloud is that ``We only remember 10\% of what we read versus 50\% of what we see.''\ftSAzOne - driving their passion for creating visual software.
\ftSAzOneText

The developers envisioned the audience for their product being customers on a global scale, thus needing to ensure that their software was scalable and able to handle a high volume of throughput. Instead of using the limited funds that were available to purchase and maintain a server farm, a cloud computing solution was chosen instead. By utilising the Windows Azure platform, Sharpcloud was able to leverage the scalability and security measures that were available in order to market their product quickly and in a cost effective manner. Choosing to migrate to the cloud rather than maintaining own servers is now saving Sharpcloud U.S.\$400,000 to \$500,000 a year.

Sharpcloud developers were able to make use of the Windows Azure development fabric service which was installed on user machines. This allowed all development to be performed locally, using the cloud only for undertaking product testing. This saved the startup company a significant sum of money, as it was not paying usage fees for development time to be performed on the cloud.

Sarim Khan, the Chief Executive Officer and Co-Founder of Sharpcloud explains how the Windows Azure platform has aided his company. ``We wouldn't exist if we had to build out of this level of server capability ourselves. Windows Azure makes it possible for us to scale the service as needed, using - and paying - only for what we need.''\ftSAzTwo
\ftSAzTwoText

\section{Amazon EC2 -- FourSquare}
Foursquare is a location based service that enables people to share their current location with their friends. Founded in 2009, Foursquare was a pioneer in location based services and has grown to more than 20 million users, with over 5 million ``check-ins'' per day (when a user opens the Foursquare app on their iPhone / Android and clicks ``check in'', sharing their current location with their friends), and with more than 750,000 partner businesses\ftSAmOne.
\ftSAmOneText

Foursquare uses AWS to perform data analysis on hundreds of millions of events generated by its application servers. Foursquare utilises AWS’s implementation of Hadoop Map-Reduce, ``Amazon Elastic Map Reduce'', which enables dynamic cluster resizing enabling foursquare to match the size of their cluster with the workload size. Flexibility is a key benefit of this arrangement, as it means that data scientists, analysts and engineers can spin up their own clusters to perform on-demand analysis without having to worry about capacity being available. This flexibility translates into massive cost savings for foursquare over hosting their own servers capable of meeting peak demand.

Since foursquare has a ``base load'' of activity that they require, they purchased \$1 million worth of ``reserved instances'', which provides them with guaranteed access to a number of instances at a reduced cost which reduced their AWS bill by 35\%. 

Mattew Rathbone, a software engineer explains the benefits of AWS over hosting servers internally: ``By expanding our clusters with Reserved Instances and On-Demand Instances, plus the Amazon EC2 price reductions, we have reduced our analytics costs by over 50\% when compared to hosting it ourselves. Additionally, we have decreased the processing time for urgent data-analysis, all without requiring additional application development or adding risk to our analytics.''\ftSAmTwo
\ftSAmTwoText

\section{Aneka -- GoFront}

The GoFront Group was established in 1936 and is one of the 512 state run enterprises in China. 
GoFront is China's largest manufacturer of rail electric traction equipment. In addition to the production of locomotives, motor train sets and other urban transportation vehicles; GoFront also is the largest researcher into the development of rail related technologies. It has over 3000 science and technology staff of an overall employee base of over 10,000\ftSAnOne{}\ftSAnTwo{}\cite{Aneka}/
\ftSAnOneText\ftSAnTwoText 

The development of new high speed electric locomotives and other transportaion vehicles is a large part of the research that occurs at GoFront. The prototypes for new designs are developed in AutoDesk, however part of the design process requires the rendering of these protoypes into high quality 3D images using the AutoDesk rendering software, Maya. 

These images are used in the identification of potential design flaws in the prototypes. As such, images are required from many different camera angles and over 2000 frames from more than 5 different angles is required to generate the final 3D rendered image. Each frame is of such high quality, that it can take up to 2 minutes to render. This can lead to the complete set of images taking over 3 days to finish, a process which must be completed after every modification to the prototype is made. 

Through the combined effort of Manjrasoft and the GoFront IT department, Aneka was deployed to take advantage of 20 desktop machines that were being under utilised. These machines were used for nothing other than the processing of word documents, and as such were turned into a private cluster with the use of the Aneka platform. The system was setup in a simple Master-Slave configuration, with 1 master computer distributing the frames to the 20 slave computers.

A custom GUI was developed in C\# using the Aneka Task Model API, that enabled the disribution of the Maya tasks to the Aneka Grid and then collecting the rendered images. Through its use, the runtime of 3 days has been reduced to 3 hours, over 20 times faster. This was with the use of only 20 of 50 computers that were identified as suitable to being added to the Aneka Grid. Through the use of otherwise wasted cycles on existing hardware, the Aneka Platform was able to fully utilise the resources available to GoFront and drastically improved the rending time of an important part of their product development process.
