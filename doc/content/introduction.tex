\chapter{Introduction}
Over the course of the last decade outsourcing the delivery of computational resources to external companies has become more and more popular for personal, business and scientific uses. In a manner analogous to water, gas, and electricity, computing power can be treated as a resource, and charged based on individual usage~\cite{Aneka}. This delivery of computing resources as a service over the Internet is known as Cloud computing. This has a number of benefits for end users wishing to access such a resources:
\begin{itemize}
\item Resources can be accessed from anywhere; users are no longer tied down to their machine in a physical location.
\item Users do not have to maintain their own hardware, other than a machine to act as a portal to these utilities.
\item Users are able to access much greater computing resources than would otherwise be available to them.
\item Users no longer have to worry about losing data if their machine fails as it is distributed across the Cloud.
\end{itemize}

\section{Cloud Computing Architecture}
Cloud computing can deliver services to end users at a number of different levels, providing resources which are both beneficial to software developers and end users~\cite{Aneka}:
\begin{itemize}
\introdef{Software as a Service (SaaS)}{Individual applications which make use of the cloud for either storage of computing power.}
\introdef{Platform as a Service (PaaS)}{A platform for development across a distributed network of machines is provided to the end user.}
\introdef{Infrastructure as a Service (IaaS)}{Off-site hardware infrastructure is provided to a number of end users to securely access and use for their own purposes.}
\end{itemize}

In this report we will be focusing on the PaaS and IaaS services available to developers, rather than looking at SaaS solutions. 
